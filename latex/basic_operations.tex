\section{Basic Operations}
\label{sec:basic_operations}

An operation on a transducer (or acceptor) takes one or more transducers as
input and outputs a transducer. You can think of these operations as functions
on graphs. As a reminder, I'll use uppercase script letters to represent
graphs, so $\gA$ for example can represent a graph. Functions will be denoted
by lower case variables. So $f(\gA)$ is a function which takes as input a
single graph and outputs a graph.

\subsection{Closure}

The closure, sometimes called the Kleene star, is a unary function (takes a
single input) which can operate on either an acceptor or transducer. If the
sequence $\vx$ is accepted by $\gA$, then zero or more copies of $\vx$ are
accepted by the closure of $\gA$. More formally, if the language of an acceptor
is $\L(\gA)$, then the language of the closure of $\gA$ is $\{\vx^n
\mid \vx \in \L(\gA),\;\; n = 0, 1, \ldots, \}$. The notation $\vx^n$
means $\vx$ concatenated $n$ times. So $\vx^2$ is $\vx \vx$ and $\vx^0$ is the
empty string. Usually the closure of an acceptor is denoted by $^*$, as in
$\gA^*$. This is the same notation used in regular expressions.

The closure of a graph is easy to construct with the use of $\epsilon$
transitions. The language of the graph in figure~\ref{fig:fsa_pre_closure} is
the string $aba$.

\begin{figure}
    \centering
    \includegraphics[scale=\dotscale]{figures/fsa_pre_closure}
    \caption{An example of a simple acceptor with language $\{aba\}$, in other
    words $\L(\gA) = \{aba\}$.}
    \label{fig:fsa_pre_closure}
\end{figure}

The closure of the graph needs to accept an arbitrary number of copies of $aba$
including the empty string. To accept the empty string we make the start state
an accept state as well. To accept one or more copies of $aba$ we simply wire
up the old accept states to the new start state with $\epsilon$ transitions.

The closure of the graph in figure~\ref{fig:fsa_pre_closure} is shown in
figure~\ref{fig:fsa_closure}.

\begin{figure}
    \centering
    \includegraphics[scale=\dotscale]{figures/fsa_closure}
    \caption{The closure $\gA^*$ of the graph in
    figure~\ref{fig:fsa_pre_closure}. The language of the graph is $\{\epsilon,
    aba, abaaba, \ldots\}$.}
    \label{fig:fsa_closure}
\end{figure}

\begin{example}
You might notice that state $4$ in the graph in figure~\ref{fig:fsa_closure} is
not necessary. Consider an alternate construction for computing the
closure of a graph. We could have made the state $0$ into an accept state
and connected state $3$ to state $0$ with an $\epsilon$ transition, as in
the graph in figure~\ref{fig:fsa_closure_2}.

\begin{figure}
    \centering
    \includegraphics[scale=\dotscale]{figures/fsa_closure_2}
    \caption{The closure $\gA^*$ of the graph in
    figure~\ref{fig:fsa_pre_closure} using an alternate, simpler construction
    which connects the accept state to the original start state with an
    $\epsilon$ transition. This construction does not work for every case.}
    \label{fig:fsa_closure_2}
\end{figure}

For the graph in figure~\ref{fig:fsa_pre_closure}, this alternate construction
works and requires fewer states and arcs. In the general case, this
construction turns every start state into an accept state instead of adding
a new start state. Give an example where this doesn't work? In other words,
give an example where the graph from this modified construction is not the
correct closure of the original graph.
\end{example}

\begin{proof}[\unskip\nopunct]
An example for which the alternate construction does not work is shown in the
graph in figure~\ref{fig:fsa_pre_closure_wrong}. The language of the graph
is $a^nb$ (any number of $a$'s followed by a $b$) and the closure is
$(a^nb)^*$, or any sequence that ends with $b$.

\begin{figure}
    \centering
    \includegraphics[scale=\dotscale]{figures/fsa_pre_closure_wrong}
    \caption{An acceptor for which the alternate construction for the closure
    does not yield the correct result.}
    \label{fig:fsa_pre_closure_wrong}
\end{figure}

If we follow the modified construction for the closure, as in the graph in
figure~\ref{fig:fsa_closure_wrong}, then the language would incorrectly
include sequences that do not end with $b$ such as $a^*$. The graph
following the correct construction of the closure is in
figure~\ref{fig:fsa_closure_right}.

\begin{figure}
    \centering
    \includegraphics[scale=\dotscale]{figures/fsa_closure_wrong}
    \caption{The alternate construction which incorrectly computes the closure
    of the graph in figure~\ref{fig:fsa_pre_closure_wrong}.}
    \label{fig:fsa_closure_wrong}
\end{figure}


\begin{figure}
    \centering
    \includegraphics[scale=\dotscale]{figures/fsa_closure_right}
    \caption{The correct closure of the graph in
    figure~\ref{fig:fsa_pre_closure_wrong} which has the language $(a^nb)^*$, which
    is any sequence which ends with $b$.}
    \label{fig:fsa_closure_right}
\end{figure}

\end{proof}

\subsection{Union}

The union takes as input two or more graphs and produces a new graph. The
language of the resultant graph is the union of the languages of the input
graphs. More formally let $\gA_1, \ldots, \gA_n$ be $n$ graphs. The language of
the union graph is given by $\{ x \mid x \in \gA_i \textrm{ for some } i = 1,
\ldots, n \}$. I'll occasionally use the $+$ sign to denote the union, as in
$\gU = \gA_1 + \gA_2$.

Since we let a graph have multiple start states and multiple accept states, the
union is  easy to construct. A state in the union graph is a start state if it
was a start state in one of the original graphs. A state in the union graph is
an accept state if it was an accept state in one of the original graphs.

Consider the three graphs in figure~\ref{fig:union_inputs} with languages
$\{ab, aba, abaa, \ldots\}$, $\{ba\}$, and $\{ac\}$ respectively.

\begin{figure}
    \centering
    \begin{subfigure}[b]{0.8\textwidth}
        \centering
        \includegraphics[scale=\dotscale]{figures/union_1}
        \caption{}
    \end{subfigure}
    \par\bigskip
    \begin{subfigure}[b]{0.8\textwidth}
        \centering
        \includegraphics[scale=\dotscale]{figures/union_2}
        \caption{}
    \end{subfigure}
    \par\bigskip
    \begin{subfigure}[b]{0.8\textwidth}
        \centering
        \includegraphics[scale=\dotscale]{figures/union_3}
        \caption{}
    \end{subfigure}
    \caption{Three acceptors with languages $\{ab, aba, abaa, \ldots\}$,
    $\{ba\}$, and $\{ac\}$ from top to bottom, respectively.}
    \label{fig:union_inputs}
\end{figure}

Notice in the union graph in figure~\ref{fig:union} the only visual distinction
from the individual graphs is that the states are numbered consecutively from
$0$ to $8$ indicating a single graph with nine states instead of three
individual graphs. The language of the union graph is $\{ab, aba, abaa,
\ldots\} \cup \{ba\} \cup \{ac\}$.

\begin{figure}
    \centering
    \includegraphics[scale=\dotscale]{figures/union}
    \caption{The union of the three acceptors in figure~\ref{fig:union_inputs}
    with language $\{ab, aba, abaa, \ldots\} \cup \{ba\} \cup \{ac\}$.}
    \label{fig:union}
\end{figure}

\subsection{Concatenation}

Like union, concatenation produces a new graph given two or more graphs as
input. The language of the concatenated graph is the set of strings which can
be formed by any concatenation of strings from the individual graph.
Concatenation is not commutative, the order of the input graphs matters. More
formally the language of the concatenated graph is given by $\{\vx_1 \ldots
\vx_n \mid \vx_1 \in \L(\gA_1), \ldots, \vx_n \in \L(\gA_n)\}$. Occasionally, I
will denote concatenation by placing the graphs side-by-side. So $\gA_1 \gA_2$
represents the concatenation of $\gA_1$ and $\gA_2$.

The concatenated graph can be constructed from the original input graphs by
connecting the accept states of one graph to the start states of the next.
Assume we are concatenating $\gA_1, \ldots, \gA_n$. The start states of the
concatenated graph are the start states of the first graph, $\gA_1$. The accept
states of the concatenated graph are the accept states of the last graph,
$\gA_n$. For any two graph $\gA_i$ and $\gA_{i+1}$, we connect each accept
state of $\gA_i$ to each start state of $\gA_{i+1}$ with an $\epsilon$
transition.

\begin{figure}
    \centering
    \begin{subfigure}[b]{0.48\textwidth}
        \centering
        \includegraphics[scale=\dotscale]{figures/concat_1}
    \end{subfigure}
    \begin{subfigure}[b]{0.48\textwidth}
        \centering
        \includegraphics[scale=\dotscale]{figures/concat_2}
    \end{subfigure}
    \caption{The acceptor on left has language $\{ba\}$ and the acceptor on the
    right has language $\{ac, bc\}$.}
    \label{fig:concat_inputs}
\end{figure}

As an example, consider the two graphs in figure~\ref{fig:concat_inputs}. The
concatenated graph is in figure~\ref{fig:concat} and has the language $\{baac,
babc\}$.

\begin{figure}
    \centering
    \includegraphics[scale=\dotscale]{figures/concat}
    \caption{The graph is the concatenation of the two graphs in
    figure~\ref{fig:concat_inputs} and has the language $\{baac, babc\}$.}
    \label{fig:concat}
\end{figure}

\begin{example}
What is the identity graph for the concatenation function? The identity in a
binary operation is the value which when used in the operation leaves the second
input unchanged. In multiplication this would be $1$ since $c * 1 = c$ for any
real value $c$.

What is the equivalent of the annihilator graph in the concatenation function?
The annihilator in a binary operation is the value such that the operation with
the annihilator always returns the annihilator. For multiplication $0$ is the
annihilator since $c *0 = 0$ for any real value $c$.
\end{example}

\begin{proof}[\unskip\nopunct]
The graph which accepts the empty string is the identity. The graph which does
not accept any strings is the annihilator. See the
figure~\ref{fig:concat_identity_annihilator} for an example of these two
graphs.

\begin{figure}
    \centering
    \begin{subfigure}[b]{0.48\textwidth}
        \centering
        \includegraphics[scale=\dotscale]{figures/concat_identity}
        \caption{Identity}
        \label{fig:concat_identity}
    \end{subfigure}
    \begin{subfigure}[b]{0.48\textwidth}
        \centering
        \includegraphics[scale=\dotscale]{figures/concat_annihilator}
        \caption{Annihilator}
        \label{fig:concat_annihilator}
    \end{subfigure}
    \caption{The identity and the annihilator for the concatenation operation.
    The language of the identity graph is the empty string $\{\epsilon\}$. The
    language of the annihilator graph is the empty set $\{\}$.}
    \label{fig:concat_identity_annihilator}
\end{figure}

The identity graph is a single node which is both a start and accept state. The
language of the identity graph is the empty string. The annihilator graph is a
single non accepting state. The language of the annihilator graph is the empty
set. Note the subtle distinction between the language that contains the empty
string and the language that is the empty set. The former can be written as
$\{\epsilon\}$ whereas the latter is $\{\}$ (also commonly denoted by
$\varnothing$).
\end{proof}

\begin{example}
Construct the concatenation of the two graphs in
figure~\ref{fig:concat_example_inputs}.
\end{example}

\begin{proof}[\unskip\nopunct]

\begin{figure}
    \centering
    \begin{subfigure}[b]{0.48\textwidth}
        \centering
        \includegraphics[scale=\dotscale]{figures/concat_example_1}
    \end{subfigure}
    \begin{subfigure}[b]{0.48\textwidth}
        \centering
        \includegraphics[scale=\dotscale]{figures/concat_example_2}
    \end{subfigure}
    \caption{The acceptor on the right has the language $\{ba, bc\}$, the
    acceptor on the left has the language $\{a, c\}$.}
    \label{fig:concat_example_inputs}
\end{figure}

The concatenated graph is in figure~\ref{fig:concat_example}.

\begin{figure}
    \centering
    \includegraphics[scale=\dotscale]{figures/concat_example}
    \caption{The concatenation of the two graphs in
    figure~\ref{fig:concat_example_inputs} has the language $\{baa, bac, bca,
    bcc\}$.}
    \label{fig:concat_example}
\end{figure}

\end{proof}

\begin{example}
Suppose we have a list of graphs to concatenate $\gA_1, \ldots, \gA_n$ where the
$i$-th graph has $s_i$ start states and $a_i$ accept states. How many new
arcs will the concatenated graph require?
\end{example}

\begin{proof}[\unskip\nopunct]
For each consecutive pair of graphs $\gA_i$ and $\gA_{i+1}$, we need to add
$a_i * s_{i+1}$ connecting arcs in the concatenated graph. So the total
number of additional arcs is:
$$
\sum_{i=1}^{n-1} a_i * s_{i+1}.
$$
\end{proof}

\subsection{Summary}

We've seen three basic operations so far:

\begin{itemize}
    \item {\bf Closure:} The closed graph accepts any string in the input graph
        repeated zero or more times. The closure of a graph $\gA$ is denoted
        $\gA^*$.

    \item {\bf Union:} The union graph accepts any string from any of the input
        graphs. The union of two graphs $\gA_1$ and $\gA_2$ is denoted $\gA_1 +
        \gA_2$.

    \item {\bf Concatenation:} The concatenated graph accepts any string which
        can be formed by concatenating strings (respecting order) from the
        input graphs. The concatenation of two graphs $\gA_1$ and $\gA_2$ is
        denoted $\gA_1\gA_2$.

\end{itemize}

\begin{example}
Assume you are given the following individual graphs $\gA_a$, $\gA_b$, and
$\gA_c$, which recognize $a$, $b$, and $c$ respectively, as in
figure~\ref{fig:fsa_tokens}. Using only the closure, union, and
concatenation operations, construct the graph which recognizes any number
of repeats of the strings $aa$, $bb$, and $cc$. For example $aabb$ and
$bbaacc$ are in the language but $b$ and $ccaab$ are not.
\end{example}

\begin{proof}[\unskip\nopunct]

\begin{figure}
    \centering
    \begin{subfigure}[b]{0.32\textwidth}
        \centering
        \includegraphics[scale=\dotscale]{figures/fsa_a}
    \end{subfigure}
    \begin{subfigure}[b]{0.32\textwidth}
        \centering
        \includegraphics[scale=\dotscale]{figures/fsa_b}
    \end{subfigure}
    \begin{subfigure}[b]{0.32\textwidth}
        \centering
        \includegraphics[scale=\dotscale]{figures/fsa_c}
    \end{subfigure}
    \caption{The three individual automata with languages $\{a\}$, $\{b\}$, and
    $\{c\}$ from left to right, respectively.}
    \label{fig:fsa_tokens}
\end{figure}

First concatenate the individual graphs with themselves to get graphs which
recognize $aa$, $bb$, and $cc$. Then take the union of the three
concatenated graphs followed by the closure. The resulting graph is shown
in figure~\ref{fig:fsa_repeats}. The equation to compute the desired graph
is $\gA = (\gA_a\gA_a + \gA_b\gA_b + \gA_c\gA_c)^*$.

\begin{figure}
    \centering
    \includegraphics[scale=\dotscale]{figures/fsa_repeats}
    \caption{The even numbered repeats graph constructed from the individual
    token graphs using the operations $\gA = (\gA_a\gA_a + \gA_b\gA_b +
    \gA_c\gA_c)^*$.}
    \label{fig:fsa_repeats}
\end{figure}

\end{proof}
